% !TeX root = ../msc-thesis.tex
\documentclass[../msc-thesis.tex]{subfiles}

\begin{document}

\chapter{\kriging reasoning}

Metamodels are a way to represent the world in simpler terms. Think of them as 
a photograph, they do not capture the moment as whole but can represent it 
good enough. In this analogy, the moment is a complex process that it is too 
cumbersome to explain it completely in mathematical terms, and metamodels, as 
photographs, may serve the purpose of capturing the core trends of this 
process without being too unwieldy and not losing too much information.

There is a family of metamodeling methodologies, ranging from a simple linear 
regression to complex neural networks. However, this chapter will be dedicated 
to discuss \kriging surrogates.

The simplest form to represent a real world process ($y$) through a metamodel 
($\hat{y}$) and its error ($\varepsilon$) is done through \autoref{eq:surreq}.

\begin{equation}
    y(x) = \hat{y}(x) + \varepsilon
    \label{eq:surreq}
\end{equation}
\nomenclature[C]{$y$}{Function that calculates the output of a process}
\nomenclature[C]{$x$}{Input of a process, or sample}
\nomenclature[C]{$\hat{y}$}{Metamodel (approximation) of $y$}
\nomenclature[C]{$\varepsilon$}{Residuals or random noise}

The error $\varepsilon$ is associated with the unmodeled effects of the inputs 
$x$ and random noise (i.e. it cannot be explained in detail but cannot be 
ignored as well.). When using the \kriging methodology as metamodel, this 
error is assumed to be a probabilistic function of $x$, or in other words, 
this error is assumed to be \textit{not} independent and identically 
distributed. The specific probabilistic function is represented by a Gaussian 
distribution with mean zero and variance $\sigma^2$.

\begin{equation}
    \varepsilon = \varepsilon(x) \sim \mathcal{N}(0, \sigma^2)
    \label{eq:krgdist}
\end{equation}

% \begin{figure}
%     \centering
%     \def\svgwidth{\columnwidth}
%     \input{../images/correlation_explain.pdf_tex}
% \end{figure}

As from \textcite{Lophaven2002}, a \kriging metamodel is comprised of two 
parts: a polynomial regression $\mathcal{F}$ and departure function $z$ of 
stochastic nature, as can be seen in \autoref{eq:kr1}.

\begin{equation}
    \hat{y}_{l}(x)=\mathcal{F}\left(\beta_{:, l}, x\right)+z_{l}(x), 
    \quad l=1, \ldots, q
    \label{eq:kr1}
\end{equation}
\nomenclature[C]{$\mathcal{F}, f(x)$}{Polynomial regression function}
\nomenclature[C]{$z$}{Stochastic departure function}

The regression model, considered as a linear combination of $t$ functions 
($f_{j}: \mathbb{R}^{n} \rightarrow \mathbb{R}$), as defined in 
\autoref{eq:kr2}.

\begin{equation}
	\mathcal{F}\left(\beta_{:, l}, x\right) \equiv f(x)^{T} \beta_{:, l}
	\label{eq:kr2}
\end{equation}

The most common choices for $f(x)$ are polynomials with orders ranging from 
zero (constant) to two (quadratic). It is assumed that $z$ has mean zero, and 
the covariance between to given points, arbitrarily named $w$ and $x$ for 
instance, is defined by \autoref{eq:kr3}:

\begin{equation}
    \operatorname{Cov}\left[z_{l}(w), z_{l}(x)\right]=\sigma_{l}^{2} 
    \mathcal{R}\left(\theta_{l}, w, x\right), \quad l=1, \ldots, q
	\label{eq:kr3}
\end{equation}

With $\sigma_{l}^{2}$ being the process variance for the \textit{lth} response 
component, and $\mathcal{R}(\theta, w, x)$ defined as the correlation model. 
In \mtc, the correlation model used is described in \autoref{eq:kr4}.

\begin{equation}
    \mathcal{R}\left(\theta_{l}, w, x\right)=\exp \left(-\sum_{i=1}^{m} 
    \theta_{l}\left(w-x_{i}\right)^{p}\right), \quad\left(\theta_{l} \geq 0, 
    p_{l} \in[0,2]\right)
	\label{eq:kr4}
\end{equation}
\nomenclature[C]{$\sigma_{l}^{2}$}{Process variance}
\nomenclature[C]{$\theta$}{\kriging hyperparameter of variable activity}
\nomenclature[C]{$p$}{\kriging hyperparameter of correlation smoothness}

Two important concepts must be addressed at this point: The first regards the 
meaning of the hyperparameter $\theta$, being interpreted as the ``activity'' 
of variable $x$, meaning that, a low value of $\theta$ indicates that the 
points are highly correlated \cite{Alves2018}. In addition, the value of 
$\theta$ also indicates how fast the correlation goes to zero as the 
process moves in the \textit{lth} direction, as discussed by 
\textcite{Caballero2008}. The second concept regards the parameter $p$ in 
\autoref{eq:kr4}, that represents the ``smoothness'' of the correlation. As 
its value reduces, the rate of the initial correlation drops as the distance 
between $w$ and $x_i$ increases. When $p \approx 0$, there is a 
discontinuity between $Y(w)$ and $Y\left(x_{i}\right)$ \cite{Forrester2008} 
and there is no immediate correlation between the given points.

The hyperparameters $\theta$ are degrees of freedom available for optimization 
purposes, seeking the improvement of the metamodel fitness. In 
\textcite{Lophaven2002}, the optimal set of hyperparameters $\theta^*$ 
corresponds to the maximum likelihood estimation. Assuming a Gaussian process 
\cite{Lophaven2002}, the optimal values of the hyperparameters solves 
\autoref{eq:kr6}:

\begin{equation}
    \min _{\theta}\left\{\psi(\theta) \equiv|R|^{\frac{1}{m}} 
    \sigma^{2}\right\}
	\label{eq:kr6}
\end{equation}

Where $|R|$ is the determinant of the correlation matrix. The internal 
optimizer used in \textit{DACE} corresponds to a modified version of 
the \textit{Hooke \& Jeeves} method, as showed by \textcite{LophavenReport}.

As stated before, high-order data obtainment it is an obligatory step in the 
proposed methodology implemented in \mtc. Fortunately, \textcite{Lophaven2002} 
also derived expressions for Jacobian evaluation of a \kriging prediction 
(for full demonstration, consult \textcite{Lophaven2002}), given in 
\autoref{eq:kr5}:

\begin{equation}
	\hat{y}^{\prime}(x)=J_{f}(x)^{T} \beta^{*}+J_{r}(x)^{T} \gamma^{*}
	\label{eq:kr5}
\end{equation}	

The expression for Hessian evaluation was derived by \textcite{Alves2018} 
(full demonstration in appendix A of their work), and it is depicted in 
\autoref{eq:kr6}:

\begin{equation}
	\hat{y}^{\prime \prime}(x)=H_{f}(x) \beta^{*}+H_{r}(x) \gamma^{*}
	\label{eq:kr6}
\end{equation}

\Cref{eq:kr5,,eq:kr6}, differently from numeric/automatic differentiation, 
are not approximations and, instead, are analytical expressions
derived by \textcite{Lophaven2002} and \textcite{Alves2018}. Therefore, it 
is expected a reduced error when one is using these expressions, if compared 
to techniques based in numerical approximation, considering that the \kriging 
metamodel used is precise enough.

For the design of experiments part, it was decided to implement the Latin 
Hypercube Sampling (LHS) because it allows to better sample the optimization 
domain without introducing ill-conditioning in the spatial correlation matrix 
calculated by the \kriging builder.

Lastly, both the LHS function and \kriging model builder/predictor were 
implemented as a separated package in Python under the name of \textit{pydace} 
(from \textit{Python toolbox for Design and Analysis of Experiments}). This 
package is a partial code translation from the MATLAB® toolbox implemented by 
\textcite{Lophaven2002} named \textit{DACE} to the Python programming 
language. The link to the open-source code is 
\url{https://github.com/feslima/pydace}. There the reader can find a brief 
documentation on how to install and example of usage.

\end{document}