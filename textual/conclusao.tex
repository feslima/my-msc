% !TeX root = ../msc-thesis.tex
\documentclass[../msc-thesis.tex]{subfiles}

\begin{document}

\chapter{Conclusion}

The software developed(\mtc) aims to become a tool that enables an easy 
deployment of the methodology for \soc structure selection through the use 
of surrogate models. The dissertation explains the complete workflow of the 
technology implemented in the software. Also, the functionalities 
and capability of \mtc were demonstrated through 3 case studies,
showing how the tool can be used for performance enhancement in the first case 
(reduction in the energy consumption in a $CO\textsubscript{2}$ compression 
process), indirect control in the second case (minimization of nominal setpoint 
deviation of a hydrocarbon separation process) and economic plantwide control 
in the third case.

In addition, a discussion was done of good practices on how to set the 
simulations, how to specify parameters in a surrogate optimization and what to 
expect of metrics used to estimate gradients and Hessian.

All the data, example files and the \mtc source code presented 
in this work can be found at \url{https://github.com/feslima/metacontrol}. The 
tool acts a resource to the scientific community to implement, analyse or 
improve current control strategies of industrial processes.

Therefore, the author invite the readers to test and give feedback on the tool 
and methodology.

\end{document}