% !TeX root = ../msc-thesis.tex
\documentclass[../msc-thesis.tex]{subfiles}

\begin{document}

\chapter{The \soc overview}

Every  industrial process is under limitations ranging from design/safety (e.g. 
temperature or pressure which an equipment can operate, etc.), environmental 
(e.g. pollutant emissions), to quality specifications (e.g. product purity), 
and economic viability. More often than not, these constraints are applied all 
at once and can be conflicting. Therefore, it is mandatory to operate such 
processes optimally (or, at least, close to its optimal point) in order to 
attain maximum profits or keep expenses at minimum while still obeying these 
specifications.

One way to achieve this is through the application of plantwide control 
methodologies. In particular, \soc \cite{Morari1980,Skogestad2000,Alstad2009} is a 
practical way to design a control structure of a process following a criterion 
(for instance: economic, environmental, performance) considering a constant 
set-point policy \cite{Alves2018}. The SOC methodology is advantageous in 
this scenario because there is no need to reoptimize the process every time 
that a disturbance occurs.

\end{document}