% !TeX root = ../msc-thesis.tex
\documentclass[../msc-thesis.tex]{subfiles}

\begin{document}

\chapter{Introduction}

This dissertation is about an assembly of several methodologies into a software 
tool, called \mtc, which enables a fast implementation of the \soc (SOC) 
technique. This assembly consist of three major methodologies: \kriging 
metamodels, optimization through infill criteria and SOC. The dissertation is 
organized as follows:
\nomenclature[A]{SOC}{\soc}

Chapter 2 gives a brief summary of the key concepts involving SOC methodology 
and the main the reason why this research and software tool development was 
needed.

Chapter 3 presents a discussion of \kriging metamodels and its reasoning.

Chapter 4 introduces the process of constrained nonlinear optimization using 
\kriging metamodels. This process is also known as infill criteria.

Chapter 5 demonstrates how the assembly of the methodologies shown in chapters 
2, 3 and 4 are combined to form the core concept behind \mtc.

Chapter 6 is dedicated to case-studies using \mtc. In addition, there is a 
brief discussion on good practices involving the use of the software tool.

The \mtc software is publicly available at 
\url{https://github.com/feslima/metacontrol}. There, the reader can find 
instructions on how to install the open-source tool. Also, for each technique 
discussed in chapters 2, 3 and 4, there is an open-source Python package as 
result. Their links are found in their respective chapters.

\end{document}