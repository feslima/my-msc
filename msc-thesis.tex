%% Adaptado por Felipe Souza Lima em 05/02/2020
%% abtex2-modelo-trabalho-academico.tex, v-1.9.7 laurocesar
%% Copyright 2012-2018 by abnTeX2 group at http://www.abntex.net.br/ 
%%
%% This work may be distributed and/or modified under the
%% conditions of the LaTeX Project Public License, either version 1.3
%% of this license or (at your option) any later version.
%% The latest version of this license is in
%%   http://www.latex-project.org/lppl.txt
%% and version 1.3 or later is part of all distributions of LaTeX
%% version 2005/12/01 or later.
%%
%% This work has the LPPL maintenance status `maintained'.
%% 
%% The Current Maintainer of this work is the abnTeX2 team, led
%% by Lauro César Araujo. Further information are available on 
%% http://www.abntex.net.br/
%%
%% This work consists of the files abntex2-modelo-trabalho-academico.tex,
%% abntex2-modelo-include-comandos and abntex2-modelo-references.bib
%%

% ------------------------------------------------------------------------------
% ------------------------------------------------------------------------------
% abnTeX2: Modelo de Trabalho Academico (tese de doutorado, dissertacao de
% mestrado e trabalhos monograficos em geral) em conformidade com 
% ABNT NBR 14724:2011: Informacao e documentacao - Trabalhos academicos -
% Apresentacao
% ------------------------------------------------------------------------------
% ------------------------------------------------------------------------------

\documentclass[
	% -- opções da classe memoir --
	12pt,				% tamanho da fonte
	openright,	% capítulos começam em pág ímpar (insere página vazia caso preciso)
	twoside,		% para impressão em recto e verso. Oposto a oneside
	a4paper,		% tamanho do papel. 
	% -- opções da classe abntex2 --
	%chapter=TITLE,		% títulos de capítulos convertidos em letras maiúsculas
	%section=TITLE,		% títulos de seções convertidos em letras maiúsculas
	%subsection=TITLE,	% títulos de subseções convertidos em letras maiúsculas
	%subsubsection=TITLE,% títulos de subsubseções convertidos em letras maiúsculas
	% -- opções do pacote babel --
	english,			% idioma adicional para hifenização
	french,				% idioma adicional para hifenização
	spanish,			% idioma adicional para hifenização
	brazil				% o último idioma é o principal do documento
	]{abntex2}


\usepackage{thesispreamble}

% ------------------------------------------------------------------------------
% Início do documento
% ------------------------------------------------------------------------------
\begin{document}

% Seleciona o idioma do documento (conforme pacotes do babel)
\selectlanguage{english}

% Retira espaço extra obsoleto entre as frases.
\frenchspacing 

% ------------------------------------------------------------------------------
% ELEMENTOS PRÉ-TEXTUAIS
% ------------------------------------------------------------------------------
% \pretextual
\subfile{pretextual/pretext-section}

% ------------------------------------------------------------------------------
% ELEMENTOS TEXTUAIS
% ------------------------------------------------------------------------------
\textual

% ------------------------------------------------------------------------------
% Introdução (exemplo de capítulo sem numeração, mas presente no Sumário)
% ------------------------------------------------------------------------------
\subfile{textual/1-introduction}

% ------------------------------------------------------------------------------
% PARTE
% ------------------------------------------------------------------------------
\part{Preparação da pesquisa}

% ------------------------------------------------------------------------------
% Capitulo com exemplos de comandos inseridos de arquivo externo 
% ------------------------------------------------------------------------------
\subfile{textual/2-desenvolvimento}


% ------------------------------------------------------------------------------
% Capitulo 3
% ------------------------------------------------------------------------------
\subfile{textual/3-dummy}

% ------------------------------------------------------------------------------
% PARTE
% ------------------------------------------------------------------------------
\part{Referenciais teóricos}
% ------------------------------------------------------------------------------

% ------------------------------------------------------------------------------
% Capitulo de revisão de literatura
% ------------------------------------------------------------------------------
\subfile{textual/4-revisao}


% ------------------------------------------------------------------------------
% PARTE
% ------------------------------------------------------------------------------
\part{Resultados}
% ------------------------------------------------------------------------------

% ------------------------------------------------------------------------------
% primeiro capitulo de Resultados
% ------------------------------------------------------------------------------
\subfile{textual/5-resultados-1}

% ------------------------------------------------------------------------------
% segundo capitulo de Resultados
% ------------------------------------------------------------------------------
\subfile{textual/5-resultados-2}

% ------------------------------------------------------------------------------
% Finaliza a parte no bookmark do PDF
% para que se inicie o bookmark na raiz
% e adiciona espaço de parte no Sumário
% ------------------------------------------------------------------------------
\phantompart

% ------------------------------------------------------------------------------
% Conclusão
% ------------------------------------------------------------------------------
\subfile{textual/6-conclusao}

% ------------------------------------------------------------------------------
% ELEMENTOS PÓS-TEXTUAIS
% ------------------------------------------------------------------------------
\postextual
% ------------------------------------------------------------------------------

% ------------------------------------------------------------------------------
% Referências bibliográficas
% ------------------------------------------------------------------------------
\printbibliography

% ------------------------------------------------------------------------------
% Glossário
% ------------------------------------------------------------------------------
%
% Consulte o manual da classe abntex2 para orientações sobre o glossário.
%
%\glossary

% ------------------------------------------------------------------------------
% Apêndices
% ------------------------------------------------------------------------------
\subfile{postextual/apendice}

% ------------------------------------------------------------------------------
% Anexos
% ------------------------------------------------------------------------------

\subfile{postextual/anexo}

%---------------------------------------------------------------------
% INDICE REMISSIVO
%---------------------------------------------------------------------
\phantompart
\printindex
%---------------------------------------------------------------------

\end{document}
