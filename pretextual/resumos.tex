% !TeX root = ../msc-thesis.tex
\documentclass[pretext-section.tex]{subfiles}

\begin{document}

% resumo em português
\setlength{\absparsep}{18pt} % ajusta o espaçamento dos parágrafos do resumo
\begin{resumo}[Resumo]
 
  Neste trabalho é apresentado uma ferramenta computacional baseada em Python 
  que permite uma rápida implementação da metodologia de Controle 
  Auto-Otimizante (do inglês \soc) com o auxílio de modelos surrogados. A 
  dissertação mostra as possibilidades e o potêncial do \textit{software} 
  \mtc através de estudos de caso de processos já estabelecidos na indústria. 
  Como resultado, obteve-se uma análise profunda destes processos por uma 
  perspectiva de controle em escala de planta (do inglês \textit{plantwide}), 
  e também são discutidas recomendações de uso da ferramenta. Além disso, os 
  dados, exemplos e o código fonte do \textit{software} \mtc estão disponíveis 
  no link \url{https://github.com/feslima/metacontrol}.

 \textbf{Palavras-chave}: Python, \soc, \kriging, Software, Plantwide.
\end{resumo}

% resumo em inglês
\begin{resumo}[Abstract]
 \begin{otherlanguage*}{english}

  In this work, it is presented a Python based software tool that enables 
  fast implementation of a Self-Optimizing Control methodology with the 
  help of surrogate models. The dissertation outlines the potential uses of the 
  \mtc software through cases studies of well established 
  industrial processes. As a result, an in-depth analysis from a plantwide
  perspective of these processes is discussed, along with recommendations
  of use. Furthermore, the data, examples and the 
  \mtc source code shown here are available to download at
  \url{https://github.com/feslima/metacontrol}.

   \vspace{\onelineskip}
 
   \noindent 
   \textbf{Keywords}: Python, \soc, \kriging, Software, Plantwide.
 \end{otherlanguage*}
\end{resumo}

\end{document}